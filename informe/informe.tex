% Clase y configuracion de tipo de documento
\documentclass[10pt,a4paper,spanish]{article}
% Inclusion de paquetes
\usepackage{amsmath}
\usepackage{amsfonts}
\usepackage[spanish]{babel}
\usepackage[utf8]{inputenc}
\usepackage[width=15.5cm, left=3cm, top=2.5cm, right=1cm, left=2cm, height= 24.5cm]{geometry}
\usepackage{fancyhdr}
\pagestyle{fancyplain}
\usepackage{listings}
\usepackage{enumerate}
\usepackage{xspace}
\usepackage{longtable}
\usepackage{caratula}
\usepackage{mathtools}
\usepackage{caption}
\usepackage{color}

% incluye macros espec materia
%\include{algo1-cmds}

% Encabezado
\lhead{Algoritmos y Estructuras de Datos I}
\rhead{Grupo 1}
% Pie de pagina
\renewcommand{\footrulewidth}{0.4pt}
\lfoot{Facultad de Ciencias Exactas y Naturales}
\rfoot{Universidad de Buenos Aires}


\newcommand{\tab}{\-\hspace{0.5cm}}
\newcommand{\enter}{$\\[6pt]$}
\newcommand{\requiere}[2] {\tab\textbf{requiere #1}: $#2$;\\[6pt]}
\newcommand{\asegura}[2] {\tab\textbf{asegura #1}: $#2$;\\[6pt]}
\newcommand{\modifica}[1] {\tab\textbf{modifica}: $#1$;\\[6pt]}
\newcommand{\aux}[1] {\textbf{aux #1}}

\begin{document}

% Datos de caratula
\materia{Algoritmos y Estructuras de Datos I}
\titulo{Trabajo Práctico Número 2}
%\subtitulo{}
\grupo{Grupo: 1}

\integrante{Ciruelos Rodríguez, Gonzalo}{063/14}{gciruelos@dc.uba.ar}
\integrante{Gatti, Mathias}{477/14}{mathigatti@gmail.com}
\integrante{Rabinowicz, Lucía}{105/14}{lu.rabinowicz@gmail.com}
\integrante{Weber, Andres}{923/13}{herr.andyweber@gmail.com}

\maketitle

% Para crear un indice
%\tableofcontents

% Forzar salto de pagina
\clearpage

\section{Observaciones}

	\begin{enumerate}
		\item a
	\end{enumerate}

% Otro salto de pagina
% \newpage

\section{Especificación}

\subsection{posicionesMasOscuras}
\enter \textbf{problema posicionesMasOscuras} ($im:Imagen$) = $result : [\langle\mathbb{Z},\mathbb{Z}\rangle]$ \{\enter
\asegura{}{mismos (result, [(i,j)| i \leftarrow [0..ancho(im)), j \leftarrow [0..alto (im)), \ esMinimo(sumaColor (color (im,i,j)), im)])}
\tab\aux{sumaColor}(p:Pixel):$\mathbb{Z}$ = red(p)+green(p)+blue(p);\\
\tab\aux{esMinimo}(s : $\mathbb{Z}$, im : Imagen) : Bool = $(\forall i \leftarrow [0..ancho(im)), \forall j \leftarrow [0..alto (im))) \ s \geq sumaColor(color (im,i,j))$
%\tab\aux{colorMinimo}(i:Imagen):$\mathbb{Z}$ = $min([sumaColor(color(i,x,y))|x\leftarrow [0..ancho(i)),i\leftarrow [0..alto(i))])$; \\
%\tab\aux{min} $(l:[\mathbb{Z}]):\mathbb{Z} =[l_i|(\forall i,j \leftarrow [0..|l|)) l_i \leq l_j]_0$;\\
\}

\subsection{top10}
\enter \textbf{problema top10} ($g:Galeria$)=result:$[Imagen]$ \ {\enter
\asegura{}{0 \leq |result| \leq 10}
\asegura{}{|imagenes(g)| \leq 10 \implies mismos(result, imagenes(g))}
\asegura{}{(\forall h \leftarrow result) h\in imagenes(g) \land esTop10(h,g)}
\asegura{}{ordenadaDecreciente([votos(g,result_i)|i\leftarrow[0..|result|)])}
\tab\aux{esTop10} $(h : Imagen, g : Galeria) : Bool$ = $|[ im | im \leftarrow imagenes(g), votos(g, im) < votos(g,h) ]| < 10$ \\
\tab\aux{ordenadaDecreciente}$(l:[\mathbb{Z}]): Bool$ = $(\forall i,j \leftarrow [0..|l|), i\geq j)l_j \geq l_i$\\
\}

\subsection{laMasChiquitaConPuntoBlanco}
\enter \textbf{problema laMasChiquitaConPuntoBlanco} $(g:Galeria)$= $result:Imagen$ \{\enter
\requiere{existeImagenConPuntoBlanco}{(\exists h \leftarrow imagenes(g)) tieneBlanco (h)}
\asegura{}{tieneBlanco(result)}
\asegura{esLaMasChica}{(\forall j \leftarrow imagenesConBlanco(g)) ancho(result)*alto(result) \leq ancho(j)*alto(j) }
\tab\aux{imagenesConBlanco} $(g:Galeria):[Imagen]$=$[H|H \leftarrow imagenes (g), tieneBlanco (H)]$\\
\tab\aux{tieneBlanco}$(i:Imagen):Bool$=$(\exists x \leftarrow [0..ancho(i)),y \leftarrow [0..alto(i))) sumaColor (color(i,x,y))==255*3$\\
\tab\aux{sumaColor}$(p:Pixel): \mathbb{Z}$ = $red(p)+green(p)+blue(p)$\\
\}

\subsection{agregarImagen}
\enter \textbf{problema agregarImagen} $(g:Galeria,i:Imagen)$ \{ \enter
\requiere{}{i\notin imagenes(g)}
\modifica{g}
\asegura{agregaImagen}{mismos (imagenes(g),imagenes(pre(g))++[i])}
\asegura{}{votos(g,i)==0}
\asegura{noCambiaVotos}{(\forall h \leftarrow imagenes(pre(g))) votos(g,h) == votos(pre(g), h)}
\}

\subsection{votar}
\enter \textbf{problema votar} $(g:Galeria,i:Imagen)$ \{ \enter
\requiere{}{i\in imagenes(g)}
\modifica{g}
\asegura{noCambiaImagenes}{mismos(imagenes(g),imagenes(pre(g)))}
\asegura{noCambiaVotosDelResto}{(\forall h \leftarrow imagenes(g), h\neq i) votos(g,h)==votos(pre(g),h)}
\asegura{}{votos(g,i)==votos(pre(g),i)+1}
\}

\subsection{eliminarMasVotada}
\enter \textbf{problema eliminarMasVotada} $(g:Galeria)$ \{ \enter
\requiere{}{|imagenes(g)| > 0}
\modifica{g}
\asegura{eliminaUnaImagen}{|imagenes (pre(g))| == |imagenes(g)|+1}
\asegura{eliminaLaMasVotada}{(\forall h \leftarrow imagenes(pre(g))) \\
\indent \indent h \notin imagenes(g) \implies (\forall j \leftarrow imagenes(pre(g))) votos(pre(g), h) \geq votos(pre(g), j)}
\asegura{noCambiaVotos}{(\forall h \leftarrow imagenes(g) , h \in imagenes(pre(g))) votos(pre(g), h) == votos(g, h)}
\}



\section{Implementacion}
\renewcommand\lstlistingname{Código fuente}
\renewcommand\lstlistlistingname{Códigos fuente}

\definecolor{darkgreen}{RGB}{0,100,0}
\definecolor{darkblue}{RGB}{0,0,127}
\definecolor{orange}{RGB}{255,69,0}
\definecolor{grey}{RGB}{63,63,63}
\lstset{language=C++,
        basicstyle=\ttfamily\footnotesize,
        showstringspaces=false,
        numbers=left,
        numberstyle=\scriptsize,
        keywordstyle=\color{darkblue},
        commentstyle=\color{darkgreen},
        stringstyle=\color{orange},
        identifierstyle=\color{black},
        numberstyle=\tiny\color{grey}, 
}
%\lstinputlisting[caption=pixel.cpp]{../src/pixel.cpp}
%\lstinputlisting[caption=imagen.cpp]{../src/imagen.cpp}
%\lstinputlisting[caption={galeria\_imagenes.cpp}]{\detokenize{../src/galeria_imagenes.cpp}}
%\lstinputlisting[caption=main.cpp]{../src/main.cpp}


\section{Demostraciones}

\subsection{Invariante de Representación}
\enter
InvRep (imp: ClaseGaleriaImagen):\\
\indent$|imp.imagenes| ==  |imp.votos| \ \land$ \\
\indent$(\forall v \leftarrow votos)\ v \geq 0 \ \land$ \\
\indent$(\forall i,j \leftarrow [0..|imp.imagenes|, i \neq j)\  imp.imagenes[i] \neq imp.imagenes[j]\ \land$ \\
\indent$(\forall i \leftarrow [0..|Imp.votos|-1))\ votos[i+1] \geq votos[i]$ \\

\subsection{Función de Abstracción}
\enter
abs (imp: ClaseGaleriaImagen, esp: Galeria):\\
\indent$mismos(imagenes(esp),imp.imagenes) \ \land$ \\
\indent$(\forall h \leftarrow imagenes(esp), \forall i \leftarrow [0..|imp.imagenes|), imp.imagenes[i] == h)\ imp.votos[i]== votos(esp,h)$ \\

\subsection{Correctitud del Código}
\enter
\textbf{\textcolor{darkgreen}{void GaleriaImagenes :: eliminarMasVotada () \{}} \\

//estado 1; \\
\indent//vale $InvRep(this);$ \\ 
\indent//vale $|imagenes|>0;$ \\

\textbf{\textcolor{darkgreen}{imagenes . pop\_back ();}} \\

//estado 2; \\
\indent//vale $|imagenes| == |imagenes@1| - 1;$ \\
\indent//vale $(\forall  i \leftarrow  [0..|imagenes@1|-1)) imagenes[i] == imagenes@1[i]; $\\
\indent//vale $votos == votos@1;$ \\

\textbf{\textcolor{darkgreen}{votos . pop\_back ();}}\\

//estado 3;
\indent //vale $|votos| == |votos@2| - 1;$ \\
\indent //vale $(\forall  i \leftarrow  [0..|votos@1|-1)) votos[i] == votos@2[i]; $\\
\indent //vale $imagenes == imagenes@2;$\\
\\
\textbf{\textcolor{darkgreen}{\}}}
\enter 

//Queremos ver que vale InvRep(this);\\
\indent  // 1; \\
\indent //vale $imagenes == imagenes@2;$ \\
\indent //implica $|imagenes| = |imagenes@1|-1 \land |votos|==|votos@1|-1;$ \\
\indent //implica $|imagenes| == |votos| (pues |imagenes@1| == |votos@1| por InvRep); $ \\

\indent // 2; \\
\indent //vale  $( \forall v \leftarrow votos@1) v \geq 0;$ \\
\indent //vale  $votos@2 == votos@1;$ \\
\indent //vale $(\forall i \leftarrow [0..|votos@2|-1)) votos[i] == votos@2[i];$ \\
\indent //implica $(\forall i \leftarrow [0..|votos@1|-1) votos[i] == votos@1[i];$ \\
\indent //implica $ (\forall v \leftarrow votos) v \geq 0 (pues es una sublista);$ (sublista de votos@1) \\

\indent // 3;\\
\indent //vale $imagenes == imagenes@2;$\\
\indent //vale $ (\forall i \leftarrow [0..|imagenes@1|-1)) imagenes@2[i]== imagenes@1[i];$\\
\indent //implica $imagenes@1[0..|imagenes@1|-1) == imagenes;$\\
\indent //vale $(\forall i,j \leftarrow [0..|imagenes@1]) i \neq j) imagenes@1[i] \neq imagenes@1[j];$\\
\indent //implica $(\forall i,j \leftarrow [0..|imagenes]) i \neq j) imagenes[i] \neq imagenes[j];$(sublista de imagenes@1)\\

\indent // 4;\\
\indent //vale  $(\forall i \leftarrow [0..|votos@1|-1)) votos@1[i+1] \geq votos@1[i] ;$\\
\indent //vale  $votos@2 == votos@1;$\\
\indent //vale $(\forall i \leftarrow [0..|votos@2|-1)) votos[i] == votos@2[i];$\\
\indent //implica $(\forall i \leftarrow [0..|votos@1|-1) votos[i] == votos@1[i];$\\
\indent // implica $(\forall i \leftarrow [0..|votos|-1)) votos[i+1] \geq votos[i] ;$(por ser sublista de votoss@1)\\
\indent //vale InvRep(this);\\

\indent //vale InvRep(this);\\
\indent //vale abs(this,g);\\

\indent //1;\\
\indent // vale $mismos(imagenes, imagenes(g));$\\
\indent //vale $(\forall i \leftarrow [0..|imagenes|)) votos(g,imagenes[i])==votos[i];$\\
\indent //vale $|imagenes| == |imagenes|-1;$(ya que imagenes == imagenes@2)\\
\indent //implica $|imagenes|+1 == |imagenes@1|;$\\
\indent //vale $|imagenes(g)|+1 == |imagenes(pre(g))||;$ (pues son mismos)\\

\indent //2;\\ 
\indent //vale $|imagenes(g)| == |imagenes(pre(g)| - 1 \land |votos(g)| == |votos(pre(g))| - 1;$\\
\indent //vale $(\forall i \leftarrow [0..|imagenes|-1) imagenes[i] == imagenes@1[i]  \land (\forall i \leftarrow [0..|votos|-1) votos[i] == votos@1[i];$\\
\indent //implica $ imagenes(pre(g)) == imagenes(g) ++ imagenes(pre(g))[imagenes(pre(g))-1] \land votos(pre(g)) == votos(g) ++ votos(pre(g))[];$ (por mismos entre g y this)\\
\indent //vale $(\forall i \leftarrow [0..|votos@1|-1)) votos@1[i+1] \geq votos@1[i] ;$\\
\indent //vale $(\forall i \leftarrow [0..|votos|-1)) votos[i+1] \geq votos[i] ;$\\
\indent //implica $(\forall i \leftarrow [0..|votos|-1)) votos@1[|votos@1-1|] \geq votos[i];$\\
\indent //implica $(\forall h \leftarrow imagenes(pre(g))) h \neq imagenes(g) \implies (\forall j \leftarrow imagenes(pre(g))) votos(pre(g),h) \geq votos(pre(g),j);$\\

\indent //3;

\end{document}

