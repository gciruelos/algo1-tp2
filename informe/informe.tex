% Clase y configuracion de tipo de documento
\documentclass[10pt,a4paper,spanish]{article}
% Inclusion de paquetes
\usepackage{amsmath}
\usepackage{amsfonts}
\usepackage[spanish]{babel}
\usepackage[utf8]{inputenc}
\usepackage[width=15.5cm, left=3cm, top=2.5cm, right=1cm, left=2cm, height= 24.5cm]{geometry}
\usepackage{fancyhdr}
\pagestyle{fancyplain}
\usepackage{listings}
\usepackage{enumerate}
\usepackage{xspace}
\usepackage{longtable}
\usepackage{caratula}
\usepackage{mathtools}
\usepackage{caption}
\usepackage{listings}
% incluye macros espec materia
%\include{algo1-cmds}

% Encabezado
\lhead{Algoritmos y Estructuras de Datos I}
\rhead{Grupo 1}
% Pie de pagina
\renewcommand{\footrulewidth}{0.4pt}
\lfoot{Facultad de Ciencias Exactas y Naturales}
\rfoot{Universidad de Buenos Aires}


\newcommand{\tab}{\-\hspace{0.5cm}}
\newcommand{\enter}{$\\[6pt]$}
\newcommand{\requiere}[2] {\tab\textbf{requiere #1}: $#2$;\\[6pt]}
\newcommand{\asegura}[2] {\tab\textbf{asegura #1}: $#2$;\\[6pt]}
\newcommand{\modifica}[1] {\tab\textbf{modifica}: $#1$;\\[6pt]}
\newcommand{\aux}[1] {\textbf{aux #1}}


\renewcommand\lstlistingname{Código fuente}
\renewcommand\lstlistlistingname{Códigos fuente}



\begin{document}

% Datos de caratula
\materia{Algoritmos y Estructuras de Datos I}
\titulo{Trabajo Práctico Número 2}
%\subtitulo{}
\grupo{Grupo: 1}

\integrante{Ciruelos Rodríguez, Gonzalo}{063/14}{gciruelos@dc.uba.ar}
\integrante{Gatti, Mathias}{477/14}{mathigatti@gmail.com}
\integrante{Rabinowicz, Lucía}{105/14}{lu.rabinowicz@gmail.com}
\integrante{Weber, Andres}{923/13}{herr.andyweber@gmail.com}

\maketitle

% Para crear un indice
%\tableofcontents

% Forzar salto de pagina
\clearpage

% Pueden modularizar el documento incluyendo otros .tex
% \include{observaciones}

\section{Observaciones}

	\begin{enumerate}
		\item a
	\end{enumerate}

% Otro salto de pagina
% \newpage

\section{Especificación}

\subsection{posicionesMasOscuras}
\enter \textbf{problema posicionesMasOscuras} ($i:Imagen$) = $result : [\langle\mathbb{Z},\mathbb{Z}\rangle]$ \{\enter
\asegura{}{mismos (result, [(x,y)| x \leftarrow [0..ancho(i)), y \leftarrow [0..alto (i)), sumaColor (color (i,x,y))==colorMinimo (i)])}
\tab\aux{sumaColor}(p:Pixel):$\mathbb{Z}$ = red(p)+green(p)+blue(p);\\
\tab\aux{colorMinimo}(i:Imagen):$\mathbb{Z}$ = $min([sumaColor(color(i,x,y))|x\leftarrow [0..ancho(i)),i\leftarrow [0..alto(i))])$; \\
\tab\aux{min} $(l:[\mathbb{Z}]):\mathbb{Z} =[l_i|(\forall i,j \leftarrow [0..|l|)) l_i \leq l_j]_0$;\\
\}

\subsection{top10}
\enter \textbf{problema top10} ($g:Galeria$)=result:$[Imagen]$ \ {\enter
\asegura{}{0 \leq |Result| \leq 10}
\asegura{}{ if\  |imagenes(g)| \leq 10 \ then\ mismos(result, imagenes(g)) \ else\ (\forall i \leftarrow sacar(imagenes(g),result))(j\leftarrow result),votos(g,i) \leq votos(g,j)}
\asegura{}{estaOrdenadaDecreciente([votos(g,result_i)|i\leftarrow[0..|result|)])}
\tab\aux{sacar} $(L:[T],A:[T]):[T]$ = $[l|(l\leftarrow L),(\forall a \leftarrow A), l \neq a]$\\
\tab\aux{estaOrdenadaDecreciente}$(l:[\mathbb{Z}]): Bool$ = $(\forall i,j \leftarrow [0..|l|), i\geq j),l_j \geq l_i$\\
\}

\subsection{laMasChiquitaConPuntoBlanco}
\enter \textbf{problema laMasChiquitaConPuntoBlanco} $(g:Galeria)$= $result:Imagen$ \{\enter
\requiere{existeImagenConPuntoBlanco}{(\exists h \leftarrow imagenes(g)) tieneBlanco (h)}
\asegura{}{tieneBlanco(result)}
\asegura{}{(\forall j \leftarrow imagenesConBlanco(g)) ancho(result)*alto(result) \leq ancho(j)*alto(j) }
\tab\aux{imagenesConBlanco} $(g:Galeria):[Imagen]$=$[H|H \leftarrow imagenes (g), tieneBlanco (H)]$\\
\tab\aux{tieneBlanco}$(i:Imagen):Bool$=$(\exists x \leftarrow [0..ancho(i)),y \leftarrow [0..alto(i))) sumaColor (color(i,x,y))==255*3$\\
\tab\aux{sumaColor}$(p:Pixel): \mathbb{Z}$ = $red(p)+green(p)+blue(p)$\\
\}

\subsection{agregarImagen}
\enter \textbf{problema agregarImagen} $(g:Galeria,i:Imagen)$ \{ \enter
\requiere{}{i\notin imagenes(g)}
\modifica{g}
\asegura{}{mismos (imagenes(g),imagenes(pre(g))++[i])}
\asegura{}{votos(g,i)==0}
\asegura{}{(\forall h \leftarrow imagenes(g), j \leftarrow imagenes (pre(g)), h==j) votos(g,h) == votos(pre(g)), j)}
\}

\subsection{votar}
\enter \textbf{problema votar} $(g:Galeria,i:Imagen)$ \{ \enter
\requiere{}{i\in imagenes(g)}
\modifica{g}
\asegura{}{mismos(imagenes(g),imagenes(pre(g)))}
\asegura{}{(\forall h \leftarrow imagenes(g),j \leftarrow imagenes(pre(g)), h==j , h\neq i) votos(g,h)==votos(pre(g),j)}
\asegura{}{votos(g,i)==votos(pre(g),i)+1}
\}

\subsection{eliminarMasVotada}
\enter \textbf{problema eliminarMasVotada} $(g:Galeria)$ \{ \enter
\requiere{}{|imagenes(g)| \neq 0}
\modifica{g}
\asegura{}{|imagenes (pre(g))| == |imagenes(g)|+1}
\asegura{}{(\forall h \leftarrow imagenes(pre(g))) h \notin imagenes(g) \implies (\forall j \leftarrow imagenes(pre(g))) votos(pre(g), h) \geq gotos(pre(g), j)}
\asegura{}{(\forall h \leftarrow imagenes(g) , h \in imagenes(pre(g))) votos(pre(g), h) == votos(g, h)}
%\asegura{}{(\exists h \in imagenes(pre(g))) (\forall j \leftarrow imagenes(g)), h\notin imagenes(g))votos(pre(g),h) \geq votos(g,j)}
%\asegura{}{(\forall j \leftarrow imagenes(pre(g)), not(esMasVotada(pre(g),j))), j \in imagenes(g)}
%\asegura{}{(\forall i \leftarrow imagenes(g) not(esMasVotada(pre(g),i)) votos(g,i)==votos(pre(g),i)}
%\asegura{}{(\forall i \leftarrow imagenes(g), esMasVotada(i)) votos(g,i)==votos(pre(g),i)}
%\tab\aux{esMasVotada} $(g:Galeria,i:Imagen):Bool$ = $(\forall j \leftarrow imagenes(g)) votos(g,i) \geq votos(g,j)$\\
\}



\section{Implementacion}
\lstset{language=C++,
        basicstyle=\ttfamily\footnotesize,
        showstringspaces=false,
        numbers=left,
        numberstyle=\scriptsize,
        %keywordstyle=\color{blue}
        %xleftmargin=\leftmargin
}
\lstinputlisting[caption=pixel.cpp]{../src/pixel.cpp}
\lstinputlisting[caption=imagen.cpp]{../src/imagen.cpp}
\lstinputlisting[caption={galeria\_imagenes.cpp}]{\detokenize{../src/galeria_imagenes.cpp}}
\lstinputlisting[caption=main.cpp]{../src/main.cpp}

\section{Demostraciones}


\end{document}


